%!TEX root = main.tex

\section{Introduction}

\boldification{Context is <any pertinent information that is needed to complete a task>. We all, including developers intuitively know this. To help developers in their programming tasks, understanding how developers build and maintain context is important. }

\boldification{However, it is hard to have an operational definition of context, over the years two schools of thoughts have evolved -- representational and interactional}

\boldification{Representational view looks at context as tangible, delineable and stable. Researchers have adapted this view in SE by operationalizing it as context represented as artifacts, tasks or environmental factors**...But it misses this}

\boldification{However, this still leaves us with the question ``what is context?''. This leads to the view of context as an interactional problem originating from ubiquitous computing}

\boldification{Interactional view defines context as a relational property that holds between objects or activities and focuses on the dynamic nature of context}

\boldification{However, although more intuitively accurate, this is more difficult to operationalize. In this paper we want to explore how observing interactions of a programmer with artifacts within and outside the IDE and the process of solving a programming task can help us gain insight about how context is created and evolved}

Within software engineering, the term context has been used implicitly and explicitly to describe the perspective gained from all relevant information. Although we intuitively understand the notion of context, refining that into an operational framework has proved elusive. And due to this difficulty in defining context, two schools of thought have emerged -- representational context and interactional context.

The representational view of context adheres to notion that context is tangible, delineable, and stable~\cite{Dourish:2004,Schilit:1994a,Abowd:1999,Pascoe:1998}. Operationalizing this view, researchers have described context as being encoded by artifacts, tasks, and environmental factors. However, these models fail to answer the question: ``what is context?'' Can context be defined as the sum of all of its parts? Or is there other elements that add to context, but are not contained within it?
The interactional view defines context as a relational property that holds between objects or activities. This view holds that activities are not disjoint from the context, but instead allow context to arise from activities~\cite{Dourish:2004}. Because of this intertwined relationship, contextual factors must be defined dynamically for each activity, and the sequence in which they occur becomes important to the definition~\cite{Viriyakattiyaporn:2010}.

Although the interactional view of context appears to intuitively be a more accurate definition, the elements included in the definition are difficult to measure and operationalize. In this paper, we explore whether observing the interactions that occur when a programmer solves a programming task can provide insights into how context is created and evolved. 
