%%%% Proceedings format for most of ACM conferences (with the exceptions listed below) and all ICPS volumes.
\documentclass[sigconf]{acmart}
%%%% As of March 2017, [siggraph] is no longer used. Please use sigconf (above) for SIGGRAPH conferences.

%%%% Proceedings format for SIGPLAN conferences 
% \documentclass[sigplan, anonymous, review]{acmart}

%%%% Proceedings format for SIGCHI conferences
% \documentclass[sigchi, review]{acmart}

%%%% To use the SIGCHI extended abstract template, please visit
% https://www.overleaf.com/read/zzzfqvkmrfzn


\usepackage{booktabs} % For formal tables


% Copyright
%\setcopyright{none}
%\setcopyright{acmcopyright}
%\setcopyright{acmlicensed}
\setcopyright{rightsretained}
%\setcopyright{usgov}
%\setcopyright{usgovmixed}
%\setcopyright{cagov}
%\setcopyright{cagovmixed}
\usepackage[ampersand]{easylist}
\usepackage{amssymb}
\usepackage{multicol}
\usepackage{multirow}
\usepackage{enumitem}
\usepackage{tabularx}
\usepackage{rotating}
\usepackage[roman]{parnotes}
\usepackage[many]{tcolorbox}

\setlength\belowcaptionskip{3pt}
\setlength\abovecaptionskip{3pt}
\setlength\floatsep{3pt plus 1pt minus 1pt}
\setlength\textfloatsep{3pt plus 1pt minus 1pt}
\setlength\dblfloatsep{3pt plus 1pt minus 1pt}
\setlength\dbltextfloatsep{3pt plus 1pt minus 1pt}

\newcolumntype{L}{>{\hsize=.8\hsize\raggedright\arraybackslash}X}
\newcolumntype{R}{>{\raggedleft\arraybackslash}X}
\newcolumntype{C}{>{\centering\arraybackslash}X}
\newcommand\setrow[1]{\gdef\rowmac{#1}#1\ignorespaces}
\newcommand\clearrow{\global\let\rowmac\relax}
\clearrow

% This will go in the next version of parnotes.sty
\makeatletter
\def\parnoteclear{%
    \gdef\PN@text{}%
    \parnotereset
}
\makeatother

% DOI
\acmDOI{10.475/123_4}

% ISBN
\acmISBN{123-4567-24-567/08/06}

%Conference
\acmConference[CHASE'18]{11th International Workshop on Cooperative and Human Aspects of Software Engineering}{May 27, 2018}{Gothenburg, Sweden} 
\acmYear{2018}
\copyrightyear{2018}


\acmArticle{4}
\acmPrice{15.00}

% These commands are optional
%\acmBooktitle{Transactions of the ACM Woodstock conference}
%\editor{Jennifer B. Sartor}
%\editor{Theo D'Hondt}
%\editor{Wolfgang De Meuter}

\newif\ifdraft
\draftfalse

\newcommand{\boldification}[1]{\ifdraft\indent\textbf{#1}\\\indent\else\relax\fi}
\newcommand\todo[1]{{\noindent\color{red}\\\textbf{TODO:~}#1\\}}


\begin{document}
\title{Context in Programming: An investigation of How Developers Create Context}
% \titlenote{Produces the permission block, and copyright information}

\author{Souti Chattopadhyay}
\affiliation{%
  \institution{Oregon State University}
}
\email{chattops@oregonstate.edu}

\author{Nicholas Nelson}
\orcid{0002-6365-7152}
\affiliation{%
  \institution{Oregon State University}
}
\email{nelsonni@oregonstate.edu}

\author{Thien Nam}
\affiliation{%
  \institution{Oregon State University}
}
\email{namt@oregonstate.edu}

\author{McKenzie Calvert}
\affiliation{%
  \institution{Oregon State University}
}
\email{calvertm@oregonstate.edu}

\author{Anita Sarma}
\affiliation{%
  \institution{Oregon State University}
}
\email{anita.sarma@oregonstate.edu}

\renewcommand{\shortauthors}{S. Chattopadhyay et al.}


\begin{abstract}
A programming context can be defined as all the relevant information that a developer needs to complete a task. Context comprises information from different sources and programmers interpret the same information differently based on their programming goal. In fact, the same artifact may create a different context when revisited. Context, therefore, by its very nature is a ``slippery notion.''

To understand how people create context we observed six programmers engaged in exploratory programming and performed a qualitative analysis of their activities. We observe that the interactions with artifacts and a mapping of meaning from those artifacts for a programming activity determines how one creates context.
\end{abstract}

%
% The code below should be generated by the tool at
% http://dl.acm.org/ccs.cfm
% Please copy and paste the code instead of the example below. 
%
% \begin{CCSXML}
% <ccs2012>
% <concept>
% <concept_id>10003120.10003121.10003126</concept_id>
% <concept_desc>Human-centered computing~HCI theory, concepts and models</concept_desc>
% <concept_significance>300</concept_significance>
% </concept>
% <concept>
% <concept_id>10003120.10003121.10011748</concept_id>
% <concept_desc>Human-centered computing~Empirical studies in HCI</concept_desc>
% <concept_significance>300</concept_significance>
% </concept>
% <concept>
% <concept_id>10003120.10003130.10003131</concept_id>
% <concept_desc>Human-centered computing~Collaborative and social computing theory, concepts and paradigms</concept_desc>
% <concept_significance>300</concept_significance>
% </concept>
% </ccs2012>
% \end{CCSXML}

% \ccsdesc[300]{Human-centered computing~HCI theory, concepts and models}
% \ccsdesc[300]{Human-centered computing~Empirical studies in HCI}
% \ccsdesc[300]{Human-centered computing~Collaborative and social computing theory, concepts and paradigms}


\keywords{Programming context, programming behavior}


\maketitle

%!TEX root = main.tex

\section{Introduction}

\boldification{Context is <any pertinent information that is needed to complete a task>. We all, including developers intuitively know this. To help developers in their programming tasks, understanding how developers build and maintain context is important. }

\boldification{However, it is hard to have an operational definition of context, over the years two schools of thoughts have evolved -- representational and interactional}

\boldification{Representational view looks at context as tangible, delineable and stable. Researchers have adapted this view in SE by operationalizing it as context represented as artifacts, tasks or environmental factors**...But it misses this}

\boldification{However, this still leaves us with the question ``what is context?''. This leads to the view of context as an interactional problem originating from ubiquitous computing}

\boldification{Interactional view defines context as a relational property that holds between objects or activities and focuses on the dynamic nature of context}

\boldification{However, although more intuitively accurate, this is more difficult to operationalize. In this paper we want to explore how observing interactions of a programmer with artifacts within and outside the IDE and the process of solving a programming task can help us gain insight about how context is created and evolved}

Within software engineering, the term context has been used implicitly and explicitly to describe the perspective gained from all relevant information. Although we intuitively understand the notion of context, refining that into an operational framework has proved elusive. And due to this difficulty in defining context, two schools of thought have emerged -- representational context and interactional context.

The representational view of context adheres to notion that context is tangible, delineable, and stable~\cite{Dourish:2004,Schilit:1994a,Abowd:1999,Pascoe:1998}. Operationalizing this view, researchers have described context as being encoded by artifacts, tasks, and environmental factors. However, these models fail to answer the question: ``what is context?'' Can context be defined as the sum of all of its parts? Or is there other elements that add to context, but are not contained within it?
The interactional view defines context as a relational property that holds between objects or activities. This view holds that activities are not disjoint from the context, but instead allow context to arise from activities~\cite{Dourish:2004}. Because of this intertwined relationship, contextual factors must be defined dynamically for each activity, and the sequence in which they occur becomes important to the definition~\cite{Viriyakattiyaporn:2010}.

Although the interactional view of context appears to intuitively be a more accurate definition, the elements included in the definition are difficult to measure and operationalize. In this paper, we explore whether observing the interactions that occur when a programmer solves a programming task can provide insights into how context is created and evolved. 

%!TEX root = main.tex

\section{Related Work}

\boldification{Context is an foundational concept borrowed from Ubiquitous Computing. Many such research capture context along with information to for retrieval purposes [\cite{Freeman:1996}(Freeman and Gelernter 1996), \cite{Dourish:2000}Dourish2000] or to influence system behavior based on usage patterns[\cite{Cheverst:2000}(Cheverst et al. 2000)]	}

%Context within computing is an intuitive concept, which ``matters a great deal''~\cite{Abowd:1999} but is very hard to define due to its highly dynamic nature. Schilit and Theimer~\cite{Schilit:1994b} defined context in the ``context-aware'' computing domain as being various factors of a user, e.g. location, time, identity. 

Context, while an intuitive concept, is very hard to define due to its highly dynamic nature. Schilit and Theimer~\cite{Schilit:1994b} defined context in the ``context-aware'' computing domain as being various factors of a user, e.g. location, time, identity. 
 
%This definition provided sufficient detail for modeling the user in the ``context-aware'' computing domain, but absolved the notion of environment as a secondary element of the user-driven factors of context.


% Research into ubiquitous computing captured context as being essential for personalizing the information retrieval process of intelligent systems~\cite{Freeman:1996}, presenting documents in contextually meaningful ways~\cite{Dourish:2000}, and influencing systems behavior based on the usage patterns of individual users~\cite{Cheverst:2000}. 

%\boldification{Authors like [\cite{Schilit:1994b}Schillit and theimer, \cite{Brown:1995}Brown, Ryan] define context in ``context-aware'' computing domain as various factors of the user like location, time, identity etc.}
%Authors such as Schilit and Theimer~\cite{Schilit:1994b}, and Brown~\cite{Brown:1995} defined context in the ``context-aware'' computing domain as being various factors of a user; such as location, time, and identity. This definition provided sufficient detail for modeling the user in the ``context-aware'' computing domain, but absolved the notion of environment as a secondary element of the user-driven factors of context.

\boldification{More related to the software engineering field are the definitions provided by [Brown, \cite{Ward:1997}Ward, \cite{Rodden:1998}Rodden] which capture context as factors of the user's environment or application settings}

Brown~\cite{Brown:1995} proposed a definition which included the characteristics of the user, the environment and the application. However, these definitions are inclined towards static models of the factors that affect context, and assume very little interplay between them.
%More closely aligned to software engineering are the definitions of Ward~\cite{Ward:1997}, Brown~\cite{Brown:1995} which included the characteristics of the user, the environment and the application. These definitions are inclined towards static models of the factors that affect context, and assume very little interplay between them.

\boldification{Murphy has made extensive contribution to import ``context'' into building better software by trying to capture various factors like programming task(Mylar)\cite{Kersten:2006}, artifacts(Hipikat)\cite{Vcubranic:2003} and the state of the environment(Gasparic)\cite{Gasparic:2017}}

Models like Mylar~\cite{Kersten:2006} and Hipikat~\cite{Vcubranic:2003} capture various factors like the nature of programming tasks, artifacts, and the development environment. These models attempt to identify the context that directly or indirectly contributes to the programmers ability to construct meaningful software. 

\boldification{Such representational understanding of context is accurate for software systems, as softwares are inherently built to represent states. However, they only answer ?what can be used to represent context?? rather than ''what is context?''}

This kind of \textit{representational} perspective of context is accurate for modeling software systems where the state of the program is the primary focus. However, these models can only answer \emph{``what can be used to represent context?''} rather than \emph{``what is context?''}

\boldification{[Pascoe]\cite{Pascoe:1998} defines context as subset of physical and conceptual state of user. [Dey]\cite{Abowd:1999} takes the notion forward by introducing user?s emotion, attention and informational state as part of context. He points out that context can be seen as interactional instead of representational}

Pascoe~\citep{Pascoe:1998} and Abowd et al.~\cite{Abowd:1999} propose perspectives that focus on the \textit{interactional} nature of context---where context is a subset of the physical and conceptual states of the user---the user's emotions, attention, and informational states.

\boldification{This view of context is hard to operationalize. Although Spyglass [Murphy and V\_porn]\cite{Viriyakattiyaporn:2010} subscribed to this notion of context to implement an intuitive recommender, a model of context that accounts for interactions is missing}

However, \textit{interactional} context is difficult to operationalize due to the transitory nature and focus on users. Although Spyglass~\cite{Viriyakattiyaporn:2010} was developed with this view of context to create a recommendation system, a model that accounts for the interactions is still missing.

%\boldification{In this paper we take a look at if visualizing what developers do throws any light as to how programmers actually build context. Without prescribing to any particular view of context, we explore what processes shape the programmer?s perceived context}


%!TEX root = main.tex

\section{Methodology}

\boldification{We observed 5 participants at work. Out of these we chose 3 participants who represent diverse styles of programming. P1 followed Test driven development approach, P4 was a conscientious programmer and S was a tinkerer.}
We conducted a lab study in which we observed six student participants while programming. Participants P1 through P5 were given a prompt which required planning, design, and development of software that mimics the rules that govern a traffic intersection. P6 was observed programming on a real-world problem while developing an IDE so that we could extend and confirm our observations beyond the domain of ``toy systems''. Within this paper, we focus on the results from P1, P4, and P6. This set of participants was chosen to represent a diverse set of programming styles; P1 adhered to the Test-Driven Development (TDD) model, P4 followed the Design-Driven Development (DDD) model, and P6 displayed a strong affinity for tinkering~\cite{Beckwith:2006}. By focusing on these three participants we are able to better understand their context while programming. 

\boldification{P1 and P4 was given to design(or redesign?? Clarify from Nick) a traffic simulator application that would be used for educational purposes. This style of programming is similar to maintenance or debugging. P6 was working on developing an IDE, which was more exploratory in nature with just some idea about the work and lesser constraints.}

P6 was not given a prompt because we wanted examine whether our initial observations from P1 to P5 would hold in a unconstrained real-world environment. This allowed us to differentiate observations that arose due to the specifics of the given traffic prompt, and those observations that appear to be universal across programming sessions and goals. The traffic prompt given to P1 to P5 is similar to a prompt use in previous software design studies~\cite{Mangano:2012}, and asks participants to design and implement a simulator for traffic intersections that can accommodate users experimenting with different timing and traffic light signal patterns. This prompt was designed to be easily understandable but difficult to implement, in order to challenge participants and be able to observe their use of context in problem solving.

\boldification{We captured their screen for an hour. P1 was asked to think aloud. We used his verbalizations to validate our understanding of context creation, usage and deletion}
Study sessions were time-boxed for a maximum of one hour, and organized as observational lab studies~\cite{Easterbrook:2008}. Participants were allowed to use their preferred development tools during the study, and video capture of their screen was recorded. Paper and pen were provided to participants for design, sketch, and note-taking purposes, and was collected at the conclusion of the study. Only P1 used a think-aloud, which was used to gain a better understanding of their mental model of context while problem solving.

\boldification{We unitized the videos using atlas.ti based on groups of coherent activities. For each unit, we noted the programming activities the participants performed and the artifacts they accesses and in what order. The set of programming activities were large adapted from [Yi wang?s] paper.}
To analyze the study session data, we unitized the screen capture data based upon both programming activity and artifact. We defined artifacts as web-pages, program elements open and visible on screen, and external notes. Example artifacts include intersection.scala as a file open in IntelliJ IDEA, and stackoverflow.com visited in Chrome Browser. Our codeset is based upon the codes developed by Yi Wang~\cite{Wang:2017}, which include: \texttt{navigation, reading questions, searching, reading search results, processing search results, viewing web resources, coding, run, debugging,} and \texttt{idle}. We do not use the \texttt{accidents} code since determining whether fast switching between artifacts is intentional, and therefore productive, is difficult to appraise from screen capture data. We added \texttt{communication} and \texttt{documentation} to the codeset, due to the nature of the tasks found in the traffic prompt and the think-aloud included in P1.

\boldification{Three of the authors coded the videos, following the IRR process. They obtained a jaccard index of 97.3\%. They also manually noted the order in which participants accessed the artifacts}

Using this codeset, the first , third and fourth authors coded a random set of data from P1 and P6 which represented <20\% of the total study data. After reaching >80\% inter-rater reliability (97.3\% Jaccard Index) across all three authors, the remaining data was coded individually by the same three authors.

%!TEX root = main.tex

\section{Obsevations}

\begin{figure*}
\includegraphics[width=\textwidth]{figures/P1timeplot}
\caption{Your caption goes here}
\end{figure*}

\begin{figure*}
\includegraphics[width=\textwidth]{figures/P6timeplot}
\caption{Your caption goes here}
\end{figure*}

%!TEX root = ../main.tex

\begin{table*}
\caption{Programming activities and the most frequently accessed artifacts associated with them}
\begin{tabular}{lll}
\toprule
Programming Activity & Artifact type [frequency] & Artifact type frequencies for each type \\
\midrule
\multirow{3}{*}{A0 - Coding} & Code [32] & [2,6,5,4,12,2,1] \\
& Tools [12] & [1, 1,5,1,3,1] \\
& Documents [2] & [2] \\
\midrule
\multirow{2}{*}{A1 - Interaction with documents} & Documents [19] & [19] \\
& External Artifacts [20] & [13,7] \\
\midrule
\multirow{2}{*}{A2 - Navigation} & Code [9] & [0,1,1,0,5,1,1] \\
& Tool [1] & [0,0,1,0,0,0] \\
\midrule
\multirow{2}{*}{A4 - Reading Task Prompt} & Documents [22] & [22] \\
& External Artifacts [15] & [10,5] \\
\midrule
\multirow{2}{*}{A5 - Searching in web/IDE} & Code [4] & [0,0,1,1,2,0,0] \\
& Tool [1] & [0,0,1,0,0,0] \\
\midrule
\multirow{2}{*}{A6 - Reading Search Results} & Code [1] & [0,0,0,0,1,0,0] \\
& Web Resource & [1] \\
\midrule
\multirow{2}{*}{A7 - Processing Search Results} & Code [2] & [0,0,0,0,2,0,0] \\
& Web Resource & [1] \\
\midrule
\multirow{2}{*}{A8 - Viewing Web Resource} & Code [1] & [0,0,0,0,1,0,0] \\
& Web Resource & [1] \\
\midrule
A9 - Debugging & Code [1] & [0,0,0,0,1,0,0] \\
\midrule
\multirow{2}{*}{A11 - Idle} & Documents [2] & [2] \\
& External Artifacts & [1,1] \\
\bottomrule
\end{tabular}
\end{table*}
%!TEX root = main.tex

\section{Future Directions}

\boldification{**Goal: apply IFT to create a model of how the context that is created and recommend artifacts for the context for a given programming activity}
We plan to explore the application of Information Foraging Theory (IFT)~\cite{Jin:2017,Fleming:2013} to model how context is created during programming, and to develop recommendations of artifacts that are relevant to a given programmer and a programming activity.

\boldification{IFT can be applied, since information seeking and cost-benefit analysis of which artifact is actually useful/worth pursuing}
IFT is applicable to programmer context since information seeking and cost-benefit analysis are central concerns when examining an artifact for usefulness (or value) to the overarching context of a programming activity.

\boldification{***IFT also has looked at goals...which is really what guides the interactions with artifacts for a given  programming activity (give examples)}
[RINI or ANITA: Would you write towards this boldification. It is outside of my depth of knowledge regarding IFT. The citation is here: \cite{Piorkowski:2015}]

\boldification{More work is needed on:
**Mapping
**sense-making of the information to see if it fits in the context also while alluded has not been really operationalized and is needed for future studies
**memory modeling
**How people recall information, and map information to a context has also not been looked at
**recall- memory has not been explicitly looked at in the IFT MODEL..recency has been used for predicting behavior, something that needs more investigation
**externalizing/ enriching information patches
**This selection of information based on the already existing context is similar to that pointed out in the IFT Domain. Obtaining and processing information has cost associated with it. The cost varies from having to open and process and new page to just recalling information already existing in the context**
}

Based upon on initial results, more work is needed on:
\begin{itemize}
\item \textbf{Mapping}: Examining the mapping between artifacts and context. The process of making sense of information contained within artifacts to determine a fit within a particular context has been alluded to in previous work~\cite{Pirolli:2005}, but has not been examined from operationalization perspective.
\item \textbf{Memory Models}: Modeling the ways in which programmers recall information and map that information to a particular context. Memory has been examined in the IFT model for predicting information-seeking behavior~\cite{Lawrance:2010,Lawrance:2013}, but there is further work needed to investigate the effects of context.
\item \textbf{Externalizing/Enriching Information Patches}: Also examined in the domain of IFT, the selection of information based upon the already existing context incurs a cost. Information costs are associated with the effort required to obtain it, but determining whether that information previously existed in context and simply be recalled has not been explored.
\end{itemize}

\bibliographystyle{ACM-Reference-Format}
\bibliography{bibliography} 

\end{document}
