%!TEX root = main.tex

\section{Future Directions}

\boldification{**Goal: apply IFT to create a model of how the context that is created and recommend artifacts for the context for a given programming activity}
We plan to explore the application of Information Foraging Theory (IFT)~\cite{Jin:2017,Fleming:2013} to model how context is created during programming, and to develop recommendations of artifacts that are relevant to a given programmer and a programming activity.

\boldification{IFT can be applied, since information seeking and cost-benefit analysis of which artifact is actually useful/worth pursuing}
IFT is applicable to programmer context since information seeking and cost-benefit analysis are central concerns when examining an artifact for usefulness (or value) to the overarching context of a programming activity.

\boldification{***IFT also has looked at goals...which is really what guides the interactions with artifacts for a given  programming activity (give examples)}
\boldification{[RINI or ANITA: Would you write towards this boldification. It is outside of my depth of knowledge regarding IFT. The citation is here: \cite{Piorkowski:2015}]}

\boldification{More work is needed on:
**Mapping
**sense-making of the information to see if it fits in the context also while alluded has not been really operationalized and is needed for future studies
**memory modeling
**How people recall information, and map information to a context has also not been looked at
**recall- memory has not been explicitly looked at in the IFT MODEL..recency has been used for predicting behavior, something that needs more investigation
**externalizing/ enriching information patches
**This selection of information based on the already existing context is similar to that pointed out in the IFT Domain. Obtaining and processing information has cost associated with it. The cost varies from having to open and process and new page to just recalling information already existing in the context**
}

Based upon on initial results, more work is needed on:
\begin{itemize}[leftmargin=6pt, parsep=0pt, topsep=0pt]
\item \textbf{Mapping}: Examining the mapping between artifacts and context. The process of making sense of information contained within artifacts to determine a fit within a particular context has been alluded to in previous work~\cite{Pirolli:2005}, but has not been examined from operationalization perspective.
\item \textbf{Memory Models}: Modeling the ways in which programmers recall information and map that information to a particular context. Memory has been examined in the IFT model for predicting information-seeking behavior~\cite{Lawrance:2010,Lawrance:2013}, but there is further work needed to investigate the effects of context.
\item \textbf{Externalizing/Enriching Information Patches}: Also examined in the domain of IFT, the selection of information based upon the already existing context incurs a cost. Information costs are associated with the effort required to obtain it, but determining whether that information previously existed in context and simply be recalled has not been explored.
\end{itemize}