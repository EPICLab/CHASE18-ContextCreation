%!TEX root = main.tex

\section{Related Work}

\boldification{Context is an foundational concept borrowed from Ubiquitous Computing. Many such research capture context along with information to for retrieval purposes [\cite{Freeman:1996}(Freeman and Gelernter 1996), \cite{Dourish:2000}Dourish2000] or to influence system behavior based on usage patterns[\cite{Cheverst:2000}(Cheverst et al. 2000)]	}
Context within computing is a foundation concept borrowed from the ubiquitous computing domain. Research into ubiquitous computing captured context as being essential to personalize the information retrieval and presentation process~\cite{Freeman:1996}

\boldification{Authors like [\cite{Schilit:1994b}Schillit and theimer, \cite{Brown:1995}Brown, Ryan] define context in ``context-aware'' computing domain as various factors of the user like location, time, identity etc.}

\boldification{More related to the software engineering field are the definitions provided by [Brown, \cite{Ward:1997}Ward, \cite{Rodden:1998}Rodden] which capture context as factors of the user's environment or application settings}

\boldification{Murphy has made extensive contribution to import ``context'' into building better software by trying to capture various factors like programming task(Mylar)\cite{Kersten:2006}, artifacts(Hipikat)\cite{Vcubranic:2003} and the state of the environment(Gasparic)\cite{Gasparic:2017}}

\boldification{Such representational understanding of context is accurate for software systems, as softwares are inherently built to represent states. However, they only answer ?what can be used to represent context?? rather than ''what is context?''}

\boldification{[Pascoe]\cite{Pascoe:1998} defines context as subset of physical and conceptual state of user. [Dey]\cite{Abowd:1999} takes the notion forward by introducing user?s emotion, attention and informational state as part of context. He points out that context can be seen as an interactional instead of representational}

\boldification{This view of context is hard to operationalize. Although Spyglass [Murphy and V\_porn]\cite{Viriyakattiyaporn:2010} subscribed to this notion of context to implement an intuitive recommender, a model of context that accounts for interactions is missing}

\boldification{In this paper we take a look at if visualizing what developers do throws any light as to how programmers actually build context. Without prescribing to any particular view of context, we explore what processes shape the programmer?s perceived context}

